\lstdefinestyle{yaml}{
     basicstyle=\color{red}\footnotesize,
     rulecolor=\color{black},
     string=[s]{'}{'},
     stringstyle=\color{red},
     comment=[l]{:},
     commentstyle=\color{black},
     morecomment=[l]{-}
 }
 
\chapter{Proof Of Concept} \label{ch:proofOfConcept}

This chapter explores the Proof of Concept (PoC) phase within the context of the thesis, aiming to validate the feasibility and efficacy of implementing SDV technologies. The main objective of this chapter is to translate theoretical concepts into concrete results, demonstrating the practical application of SDV in real-world situations. Through the PoC, we aim to confirm the fundamental principles and features of SDV, including its potential impact on vehicle performance, user experience, and overall safety.

The multifaceted nature of SDV requires a structured approach to its implementation, taking into account factors such as standardized hardware, cloud integration, and over-the-air (OTA) updates. To achieve this, a PoC was designed to address these components individually and holistically, ensuring a seamless integration that aligns with the envisioned paradigm shift in automotive manufacturing. Furthermore, this chapter aims to demonstrate the collaborative efforts with industry-leading technologies and platforms, highlighting the strategic partnerships forged with key players in the automotive and software development sectors. By aligning with renowned entities, the PoC aims to leverage their expertise, technologies, and frameworks, thereby enhancing the robustness and scalability of the SDV ecosystem.

Test and Validation are the concluding phases of this chapter, where the Proof of Concept is subjected to real-world scenarios. A demonstration involving a Raspberry Pi (RPi) serves as a tangible validation of the implemented SDV functionalities. This section serves as the litmus test, affirming the seamless orchestration of SDV within the envisioned architecture.

The exploration of the POC begins by detailing the services and technologies offered by Amazon Web Services (AWS) in the IoT and automotive fields that are essential for project implementation.

\section{AWS Used Services}
\section{Design}
\subsection{Architecture}

\section{Implementation}
\subsection{Code}
\subsection{Tools}

\section{Test and Validation}
\subsection{RPi demo}
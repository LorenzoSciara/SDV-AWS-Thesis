\sommario

%In the dynamic landscape of the automotive industry, the imperative to equip vehicles with flexible hardware and upgradable software is a key priority, both in terms of efficiency and ensuring the safety of the vehicle itself.

%Cloud technology addresses these needs by providing virtually unlimited resources to ensure reliability and security through cost-effective pay-per-use services for companies.

%The work of this project, carried out in collaboration with the company Storm Reply, is dedicated to the implementation of a platform focused on the Software Defined Vehicle (SDV) paradigm, leveraging the comprehensive services provided by Amazon Web Services (AWS) and aiming to create a convergence point between the edge device and the cloud environment, ensuring an advanced and secure experience for the end user.

%A Software Defined Vehicle is characterised as a vehicle that primarily or entirely manages its operations, incorporates additional functionality and enables new features through software. The concept is based on the synergistic use of cloud technology for server-side operations such as updates, coupled with general-purpose hardware for vehicle-side functions. This technological integration significantly enhances vehicle security from multiple perspectives, including Human Safety Critical Security and Intrinsic Software Security.

%The thesis starts with an introduction on the objectives of the project and goes on to provide a broad overview of the state of the art methodologies currently used in the cloud space, specifically related to the automotive industry.

%Secondly, the entire stack for the development, maintenance and deployment of software for connected vehicles is examined in detail, along with techniques for secure communication between the vehicle and the cloud.

%Finally, a real-world project is examined, in which a sample infrastructure for maintaining and deploying code, as well as analysing data from the simulated vehicle, was created using AWS Cloud Services.

%The results of this project provide a complete illustration of a Software Defined Vehicle Platform (SDVP), highlighting the successful integration of flexible hardware, extensible software, and secure cloud-based services. This reframing emphasises the evolving landscape of SDV technology and aligns with current trends and priorities within the automotive sector.

This thesis project delves into the analysis of contemporary connected vehicle platforms, focusing on the benefits and challenges associated with these advanced solutions and emphasising aspects of safety and flexibility. 
A key trend in the current automotive sector is the prospect of transforming the car from a hardware-focused product to a software-driven device. The technology of choice for leading software development and production companies driving this change is the Software Defined Vehicle (SDV).

The primary objective of the thesis is to apply this paradigm to the development of a simulator for a vehicle control unit responsible for collecting telemetric data from the vehicle.
The implementation of the simulator involves an in-depth analysis of the drawbacks of the automotive software production industry and the advantages of the Software Defined Vehicle solution. The simulator implementation also includes the creation of a scaled-down version of a connected vehicle platform, storage infrastructure and example application.

Using the Amazon Web Services (AWS), an environment in the cloud is established for the development of the necessary software for the operation of the vehicle control unit.
Development of the vehicle control unit simulator is carried out, including client connectivity to interact with the cloud platform, telemetry generation, logic for remote operations, and optional applications. The final phase involves testing the simulator on compatible hardware to validate its functionality and performance.

The successful completion of this project in collaboration with Storm Reply, not only highlights the potential of the software-defined vehicle paradigm as a leading force in the future of the automotive sector, but also explores the economic, safety and security benefits associated with its adoption, paving the way for significant progress in the field and ensuring an advanced and safe end-user experience.

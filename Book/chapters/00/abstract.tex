\sommario

In an ever-evolving automotive context, the need for vehicles equipped with flexible hardware and upgradable software is an essential priority, both in terms of efficiency and ensuring the safety of the vehicle itself. 
Cloud technology comes to the aid of these requirements by providing virtually unlimited resources to ensure reliability and security through cost-effective pay-per-use services for businesses. 

The work of this project, conducted in collaboration with the company Storm Reply, focuses on the implementation of a platform dedicated to connected vehicles, utilizing the services provided by Amazon Web Services (AWS). 
The main objective is to create a convergence point between the edge device and the cloud environment, ensuring an advanced and secure experience for end-users. 

The thesis begins with an introduction about the goals of the project and proceeds providing a broad overview of state-of-the-art methodologies currently employed in the cloud field, specifically related to the automotive industry.\\
Secondly, it delves deeply into the entire stack for the development, maintenance, and deployment of software intended for connected vehicles, along with techniques for secure communication between the vehicle and the cloud.\\
Subsequently, a real project is examined in which an example of infrastructure for code maintenance and deployment, as well as the analysis of data from the simulated vehicle, was created using AWS cloud services.

The outcomes obtained from this project offer a comprehensive exemplification of a Connected Vehicle Platform (CVP), demonstrating the successful integration of flexible hardware, upgradable software, and secure cloud-based services.